\documentclass{sigchi}

% Use this section to set the ACM copyright statement (e.g. for
% preprints).  Consult the conference website for the camera-ready
% copyright statement.

% Copyright
\CopyrightYear{2017}
%\setcopyright{acmcopyright}
\setcopyright{acmlicensed}
%\setcopyright{rightsretained}
%\setcopyright{usgov}
%\setcopyright{usgovmixed}
%\setcopyright{cagov}
%\setcopyright{cagovmixed}
% DOI
\doi{http://dx.doi.org/10.475/123_4}
% ISBN
\isbn{123-4567-24-567/08/06}
%Conference
\conferenceinfo{CHI'16,}{May 07--12, 2016, San Jose, CA, USA}
%Price
\acmPrice{\$15.00}

% Use this command to override the default ACM copyright statement
% (e.g. for preprints).  Consult the conference website for the
% camera-ready copyright statement.

%% HOW TO OVERRIDE THE DEFAULT COPYRIGHT STRIP --
%% Please note you need to make sure the copy for your specific
%% license is used here!
% \toappear{
% Permission to make digital or hard copies of all or part of this work
% for personal or classroom use is granted without fee provided that
% copies are not made or distributed for profit or commercial advantage
% and that copies bear this notice and the full citation on the first
% page. Copyrights for components of this work owned by others than ACM
% must be honored. Abstracting with credit is permitted. To copy
% otherwise, or republish, to post on servers or to redistribute to
% lists, requires prior specific permission and/or a fee. Request
% permissions from \href{mailto:Permissions@acm.org}{Permissions@acm.org}. \\
% \emph{CHI '16},  May 07--12, 2016, San Jose, CA, USA \\
% ACM xxx-x-xxxx-xxxx-x/xx/xx\ldots \$15.00 \\
% DOI: \url{http://dx.doi.org/xx.xxxx/xxxxxxx.xxxxxxx}
% }

% Arabic page numbers for submission.  Remove this line to eliminate
% page numbers for the camera ready copy
% \pagenumbering{arabic}

% Load basic packages
\usepackage{balance}       % to better equalize the last page
\usepackage{graphics}      % for EPS, load graphicx instead 
\usepackage[T1]{fontenc}   % for umlauts and other diaeresis
\usepackage{txfonts}
\usepackage{mathptmx}
\usepackage[pdflang={en-US},pdftex]{hyperref}
\usepackage{color}
\usepackage{booktabs}
\usepackage{textcomp}
\usepackage{listings}
\definecolor{lightgray}{rgb}{.9,.9,.9}
\definecolor{darkgray}{rgb}{.4,.4,.4}
\definecolor{purple}{rgb}{0.65, 0.12, 0.82}

\lstdefinelanguage{JavaScript}{
  keywords={typeof, new, true, false, catch, function, return, null, catch, switch, var, if, in, while, do, else, case, break},
  keywordstyle=\color{blue}\bfseries,
  ndkeywords={class, export, boolean, throw, implements, import, this},
  ndkeywordstyle=\color{darkgray}\bfseries,
  identifierstyle=\color{black},
  sensitive=false,
  comment=[l]{//},
  morecomment=[s]{/*}{*/},
  commentstyle=\color{purple}\ttfamily,
  stringstyle=\color{purple}\ttfamily,
  morestring=[b]',
  morestring=[b]"
}

\lstset{
   language=JavaScript,
   extendedchars=true,
   basicstyle=\footnotesize\ttfamily,
   showstringspaces=false,
   showspaces=false,
   numbers=left,
   numberstyle=\footnotesize,
   numbersep=9pt,
   tabsize=2,
   breaklines=true,
   showtabs=false,
   captionpos=b
}

% Some optional stuff you might like/need.
\usepackage{microtype}        % Improved Tracking and Kerning
% \usepackage[all]{hypcap}    % Fixes bug in hyperref caption linking
\usepackage{ccicons}          % Cite your images correctly!
% \usepackage[utf8]{inputenc} % for a UTF8 editor only

% If you want to use todo notes, marginpars etc. during creation of
% your draft document, you have to enable the "chi_draft" option for
% the document class. To do this, change the very first line to:
% "\documentclass[chi_draft]{sigchi}". You can then place todo notes
% by using the "\todo{...}"  command. Make sure to disable the draft
% option again before submitting your final document.
\usepackage{todonotes}

% Paper metadata (use plain text, for PDF inclusion and later
% re-using, if desired).  Use \emtpyauthor when submitting for review
% so you remain anonymous.
\def\plaintitle{ExploX - Seminar Project Proposal}
\def\plainauthor{Kevin M\"uller, Marc Rupp, Lukas Strobel, Xueting Li}
\def\emptyauthor{}
\def\plainkeywords{}

% llt: Define a global style for URLs, rather that the default one
\makeatletter
\def\url@leostyle{%
  \@ifundefined{selectfont}{
    \def\UrlFont{\sf}
  }{
    \def\UrlFont{\small\bf\ttfamily}
  }}
\makeatother
\urlstyle{leo}

% To make various LaTeX processors do the right thing with page size.
\def\pprw{8.5in}
\def\pprh{11in}
\special{papersize=\pprw,\pprh}
\setlength{\paperwidth}{\pprw}
\setlength{\paperheight}{\pprh}
\setlength{\pdfpagewidth}{\pprw}
\setlength{\pdfpageheight}{\pprh}

% Make sure hyperref comes last of your loaded packages, to give it a
% fighting chance of not being over-written, since its job is to
% redefine many LaTeX commands.
\definecolor{linkColor}{RGB}{6,125,233}
\hypersetup{%
  pdftitle={\plaintitle},
% Use \plainauthor for final version.
%  pdfauthor={\plainauthor},
  pdfauthor={\emptyauthor},
  pdfkeywords={\plainkeywords},
  pdfdisplaydoctitle=true, % For Accessibility
  bookmarksnumbered,
  pdfstartview={FitH},
  colorlinks,
  citecolor=black,
  filecolor=black,
  linkcolor=black,
  urlcolor=linkColor,
  breaklinks=true,
  hypertexnames=false
}

% create a shortcut to typeset table headings
% \newcommand\tabhead[1]{\small\textbf{#1}}

% End of preamble. Here it comes the document.
\begin{document}

\title{\plaintitle}

\numberofauthors{4}
\author{%
  \alignauthor{Kevin M\"uller\\
    \affaddr{Saarbr\"ucken, Germany}\\
    \email{s9kvmuel@stud.uni-saarland.de}}\\
  \alignauthor{Marc Rupp\\
    \affaddr{Saarbr\"ucken, Germany}\\
    \email{s9mcrupp@stud.uni-saarland.de}}\\
  \alignauthor{Lukas Strobel\\
    \affaddr{St. Ingbert, Germany}\\
    \email{s8lustro@uni-saarland.de}}\\
  \alignauthor{Xueting Li\\
    \affaddr{Saarbr\"ucken, Germany}\\
    \email{ding14552@gmail.com}}\\
}

\maketitle

\begin{abstract}
  In this proposal, we present ExploX, an app that allows for route creation and exploration of less frequented areas.With this we want to enable the user to generate a more holistic image of their living surroundings and allow them to escape their daily routine. \textbf{}
\end{abstract}

\category{H.5.m.}{Information Interfaces and Presentation
  (e.g. HCI)}{Miscellaneous} \category{See
  \url{http://acm.org/about/class/1998/} for the full list of ACM
  classifiers. This section is required.}{}{}

 
 
 \keywords{Route planning;
 Exploration of unfamiliar areas;
 Navigation;
 Ubiquitous Sports technologies; 
 Endurance sports;
 Motivation  }
 
 \begin{comment}
\section{Keywords}
 Route planning;
 Exploration of unfamiliar areas;
 Navigation;
 Ubiquitous Sports technologies; 
 Endurance sports;
 Motivation  
\end{comment}

 \begin{comment}
\section{LATEX STUFF}
\begin{align*} A Formula = \{1,2,4,7\}   \quad Y = \{3,5,6,8,9,11\} \end{align*} and the Relation \begin{align*} \mathcal{R} = \{(a,b) \in  D^2 | a \neq b \wedge c=a+b \; with\; c \in Y \} \end{align*}
A reference \cite{dorr2009gaze}
 \begin{figure}
\centering
  \includegraphics[width=0.2\columnwidth]{figures/cats}
  \caption{This is a sample figure}~\label{fig:figure2}
\end{figure}

\end{comment}


\section{Introduction}
In the last 10 years, the services such as Google Maps that enable point to point navigation and  Geocaching that enable people to explore unfamiliar areas have made great progress and bring more fun. Nowadays also a variety of services such as Strava and Komoot emerge not only navigation but also route planning for running and biking. Based on these services, we want to develop an app which could plan routes and help do personal exploration of the unseen area with the history data using the API of Strava, motivating athletes to cover more less visited parts of their environment.

Our aim here is to motivate athletes to explore more of less visited parts of cities. To accomplish this ,firstly we aim to create routes along paths around areas that the user is most unlikely to know about. The maps would be divided into two parts, the familiar area and unfamiliar area using the history data. The app would recommend person A the routes in seldom visited area but frequent visited by person B, in terms of safety. Secondly our work will allow the user to explore neglected areas knowing the current location and specific destination. So the generated route could be downloaded and preview before exploring. Thirdly, the explored areas would fade out from dark black to transparent according the the frequency of visiting.

Psychogeography is the practice of exploring the urban environment with the intention to investigate the effects on feelings and behavior. To generate a complete image of the city this is necessary, as the Psychogeographers argue. This is one of the  reason we want to create the fading map.

Taking routes of another athlete could avoid the case of illegal exploration or simply routes that are not suited for running or cycling. One has to explore parts of the environment that are normally not frequented.As our aim with this project is not to support illegal activities we do not consider the actual exploration of such areas  which is regarded as danger.

Using the social media sources and personal location history we determine areas that are less frequented and might be unknown even to citizens living in the city for a long time. The design of the ExploX application, tries to combine the ideas of the psychogeography movement from the 60s with more recent developments of the exploration and Strava communities. With this we want to enable the user to generate a more holistic image of their living surroundings and allow them to escape their daily routine. Additionally we hope to complete the users spatial memory about their city.

The paper is structured as follows. First we present the introduction and related work that led to the development of this idea. Afterwards we lay out the designing milestones, a list of equipment and resources we need and the formal requirements for this prototype. Thereafter we will present how we want to conduct the user study in order to evaluate our approach. 

\section{Related Work}
In the following section, we will examine previous work done on the topics of route planning, (urban) exploration and motivation in sports and sightseeing. While the former two topics will give us an overview over different approaches to navigation and goals in exploration, the latter will help us design our prototype in order to motivate users to explore unfamiliar areas as well as complete their training routine and become better athletes.
\subsection{Route Planning}
Route planning has become a very hot research topic in mobile HCI because of the rise of smartphones, smartwatches and other wearables. Pedestrian navigation is particularly interesting because it is much more diverse than the regular turn-by-turn navigation used in car navigation systems.\\
McGookin and Brewster have done an analysis on how runners navigate the environment and presented a novel navigation system for runners \cite{undirectedrunnernav}. Their main finding was that there are two types of running practices. \textit{Familiar location running} is characterized by loops (i.e. circuit tracks). Runners usually plan the route beforehand but only use their mental model of the environment while running. On the other hand, \textit{unfamiliar location running} usually have back runs (i.e. runners go from A to B and then the same way back to A). This is a problem, because unfamiliar location running is mainly used ''as a way to explore the environment and identify places to later visit''. Our approach will address this problem by giving the runners the opportunity to explore new areas and at the same time see as much as possible by not running the same way back again.\\
In order to provide a good running experience, the route has to be selected carefully. There are many approaches that not only use distance and time to calculate a route but also take into account other factors. Stolfi and Alba noticed that most navigators calculated the same route leading to traffic jams \cite{evolutionaryalgonav}. They proposed a system that uses traffic data to re-direct cars in order to spread traffic more evenly, leading to an overall better traffic situation. In particular, they improved travel times by 18\% and greenhouse gas emission by 14\%.\\
Katayama et al. took a similar approach and navigated visitors of events using body-worn sensors in order to avoid congestions and other problems that are difficult for event managers \cite{routeplanningbodywornsensors}.\\
There are web mapping services such as Strava (\hyperlink{http://strava.com}{www.strava.com}) and Komoot (\hyperlink{http://komoot.com}{www.komoot.com}) that also provide information about the surface of the track in order to better plan the route. For instance, cyclists can easily identify off-road streets and plan their route accordingly.\\
As an extension to those online services, Daiber et al. have proposed a concept of \textit{pioneers} to support mountaineers in their route planning. The idea is that users can select a number of friends or experienced mountaineers called \textit{pioneers}. The user will see the routes their pioneers have recently taken and can incorporate them into their own route planning \cite{followthepioneers}. \\


\subsection{Exploration}
As already identified by McGookin and Brewster, exploring the environment is an important motivational factor for runners \cite{undirectedrunnernav}. In familiar location running, the main objectives are to meet a distance, time or place target rather than enjoying the environment. However, when the athlete is in an unfamiliar location, these objectives are reversed. They find that this is mostly the case on holidays, however, we want to find out whether we can get athletes to take unfamiliar routes and explore areas of cities in which they have been living for a longer period of time as well.\\
Robinson et al. have implemented an approach where they encouraged people to explore an area by giving different haptic feedback when they can take alternative routes \cite{ididitmyway}. They could show that people were able to reach their target with only low-resolution haptic feedback and providing users with alternative path awareness is also beneficial.\\
In a similar way, O'Hara could identify discovering and exploring new places as one of the main motivations in geocaching \cite{o2008understanding}. The targets geocachers are looking for are often hidden in special places that are particularly beautiful or abandoned such as old factories or hospitals.\\
There has also been a movement called \textit{urban exploration movement} where people go to and explore abandoned places \cite{urbanexploration}. The growing interest in geocaching and urban exploration shows that people care about the environment they are living in and want to find out more about it. To support this, Quercia et al. have build a system to determine aesthetic qualities of a city \cite{aestheticslondon}. They used this data in a navigation system where users are guided through particularly beautiful, happy or quiet areas.

\subsection{Motivation and Design}
When designing our system, we must not forget about what motives athletes to do sports in the first place. It is our goal to provide a motivating way to explore the city but this alone will probably not suffice to encourage athletes to use our system in their regular training routine. We will try to incorporate several motivational factors into the design of our system.\\
Vallerand et al. have identified three psychological needs which are the reason why people take part in sports \cite{vallerand2007intrinsic}. Those needs are the need for autonomy, competence and relatedness \cite{vallerand1999integrative}. The need for competence is satisfied by giving the athletes regular success and not make them fail all the time. Some competition is good but overall the training climate should be mastery-oriented and not highly competitive. The need for relatedness is satisfied by providing the possibility for athletes to cooperate, share and do activities together. Giving the users some freedom of what they want to do satisfies their need for autonomy \cite{vallerand2007intrinsic}.\\
In a similar fashion, Ross and Iso-Ahola have identified knowledge-seeking and social interaction as the dominating motivation force in sightseeing. The bottom line from their research is that it is important to teach users something while they are exploring and at the same time give them the opportunity for social interaction \cite{DUNNROSS1991226}.

\section{Design and Implementation}
\subsection{Vertical Prototypes}
Our system is very dependent on different third party libraries, web services as well as technical capabilities of the platform. Therefore, it is wise to develop quick vertical prototypes for those parts first, before starting with the design and development of the actual software architecture. In this section we will have a quick look on the vertical prototypes we implemented for various critical functionalities.\\
Since our system performs a lot of calculations on existing route data, we had to find an efficient way to store and execute spatial queries on spatial data. Routes and activities consist of a list of ordered coordinates, represented as latitude-longitude pairs, which act as waypoints. We chose MongoDB\footnote{www.mongodb.com}, which is a database that specifically supports such coordinates and allows to execute queries such as ''Find all coordinates within 20km of a given point''. After designing our route generation algorithm in theory, we tested the relevant queries on the database in order to verify that it suits our needs. \\
A critical component of our system is the visualization of the map which includes displaying explored areas as well as single routes. We researched ways how we can display maps on a website and found Leaflet.js as a good candidate. By studying the documentation, we learned that there exist several plugins for this library. We had a look at each of the relevant libraries individually and tested them out with some sample data.\\
The main use case of our system is the generation of new cycling routes, so we had to find a service that can do that. The Open Source Routing Machine (OSRM)\footnote{www.project-osrm.org} is the most popular way to do this. We will explain this part in more detail below.

\subsection{Implementation}
The software has been implemented in JavaScript. Although we have a front-end graphical user interface that runs in the browser, the main program logic happens on the server. To achieve that, we used Node.js\footnote{www.nodejs.org} as backend JavaScript runtime. The main reasons why we think using Node.js in this context is a good idea are, first, the application is communicating a lot with other APIs by sending and receiving requests over the network. Second, we have a front-end that runs JavaScript in the browser but also third, the Node.js ecosystem has grown a lot in recent years and there is a huge number of open-source packages available, whose dependencies can easily be managed by using the Node package manager.\\
An incremental and iterative approach is very useful when designing and implementing software, which is why we deeply incorporated that paradigm into our development process. After implementing a certain feature, it went through a short phase of testing and evaluation, before we proceeded to the next feature. Also, for most of the methods and functionalities, we created unit tests that constantly ensure the correctness of our code.\\
The whole system has been designed in a modular way which allows it to be easily maintained and modified for other projects. In the following, we will explain the non-trivial parts of the implementation.
\subsubsection{User Management}
To make later evaluations and studies more easy, we decided to start with a solid user management. We defined database schemas for users and provided forms for logging in to existing accounts or signing up new users. The password is salted and hashed and stored in the database. Sessions are created such that the user stays logged in in their browser as long as the session is active. In order to use our features, users have to connect their Strava account. Therefore, we had to implement an OAuth authentication for Strava. Users who wish to log in using their Strava account are redirected to the Strava login page where they have to log in and authorize our app to access their profile data. This only has to be done once.
\subsubsection{Spatial Queries}
As a database for both regular tables with route, activity and user data as well as special geographic data we used MongoDB and Mongoose\footnote{www.mongoosejs.com} as database driver for Node.js. Each route or activity waypoint is a separate entry in the table that consists of a latitude-longitude-pair, a list of references to routes and activities that contain this point as well as an optional name for special locations (e.g. ''Home''). The schema can be seen in listing 1.
\begin{lstlisting}[caption=Geo Schema with Spatial Index]
const GeoSchema = new Schema({
  name: { type: String, default: '',
    trim: true },
  location: { type: Schema.Types.Point, 
    index: '2dsphere' },
  routes: [{ type: Schema.ObjectId, 
    ref: 'Route' }],
  activities: [{ type: Schema.ObjectId, 
    ref: 'Activity' }],
});
\end{lstlisting}
Notice that we added an index called 2dsphere on the location attribute (line 5) which allows the database to correctly execute spatial queries, considering the earth's curvature. The example query in listing 2 returns all datapoints within 1000m of the given coordinate 'coord' together with the exact distance to this coordinate added in a field called 'distance' in ascending order. We make use of this functionality in the route generation algorithm.
\begin{lstlisting}[caption=Sample Spatial Query]
Geo.aggregate([{
  $geoNear: {
    query: {},
    near: {
      type: 'Point',
      coordinates: [coord.lng, coord.lat]},
    maxDistance: 1000,
    minDistance: 0.1,
    spherical: true,
    distanceField: 'distance',
    }
  },
  { $sort: { distance: 1 }}])
.exec(callback);
\end{lstlisting}
\subsubsection{Strava API}
As described earlier, our goal is to generate new and good cycling routes. To make sure those routes are good for cycling and include interesting sections, we decided to use existing routes and segments from Strava to lead our algorithm. Moreover, the activity map should display the user's past cycling activities which we can also get form Strava. In order to get all this data, we had to implement the corresponding Strava API calls, namely getting the user's profile data, getting a user's personal routes and activities, getting the coordinates of each route and activity, getting a list of segments for a given area and getting the coordinates for each of them. Unfortunately, the API does only allow getting routes from users that have explicitly authorized our app to do so, which means we only have the routes of users available that have registered on our website. To get most of the available segments, we implemented a specialized crawler that works around the API limitations and can retrieve a good amount of Strava segments. For example, in the area of Saarbr\"ucken to St. Ingbert, we could retrieve more than 200 of the most popular segments.
\subsubsection{Map Visualization}
The map visualization is rather straight forward. As a map provider, we use Leaflet.js\footnote{www.leafletjs.com} and a map from Open Street Maps. To display the activity map, we integrated a plugin called MaskCanvas\footnote{www.github.com/domoritz/leaflet-maskcanvas } which overlays the base layer map with a canvas. We chose this canvas to have a dark and semi-transparent background. The plugin receives all the coordinates from the user's activities and renders the overlay fully transparend at those locations with a radius of 200m. We used Leaflet.js's GeoJSON markers to highlight the start and end points of the route with colored circles as well as thin lines connecting each waypoint directly as an additional guide when the actual routing is not clear (e.g. on route intersections).
\subsubsection{Routing Generation}
Lastly, the most interesting part of our system: The algorithm for generating new routes. Wherever possible the operations are implemented in an asynchronous (i.e. non-blocking) way to maximize efficiency. Basically, there are four steps that in the end lead to a list of five or less matching route suggestions.\\
1. Lower Bound Distance Filter: We get all routes and segments that are shorter than the wanted route distance from the database. Then, for each route/segment we calculate the distance from the given start point to each the start and end point. Those two measures summed up with the distance of the route/segment itself gives us the lower bound on the distance that a route that includes this route/segment would have.\\
2. Combine, Sort and Reduce: Next, routes and segments are combined such that multiple short segments would be connected and included in the final route. By doing so, we can find a new route that traverses as many segments as possible in the given distance. However, in our prototype, we only used single routes and segments and did not combine them, but the design of the algorithm as it is implemented already allows this feature. Next, the remaining combinations are sorted on the lower bound distance and filtered such that the top 20 combinations that match the wanted distance remain.\\
3. Find Route: In this step, we down-sample the route/segments that we want to include in our route such that only 25 waypoints remain. We then query the MapBox\footnote{www.mapbox.com} directions API and pass the waypoints as well as some other relevant parameters. The API will then return a route from and to the given start point that includes the given route/segment. Another distance filter removes routes where the actual distance deviates a lot from the lower bound distance (i.e. routes that are much too long).\\
4. Familiarity Filter: The last step includes filtering the found routes based on what percentage of the route is in a familiar area. To get this value, we make use of the spatial database queries by checking for each of the waypoints whether an explored point is within a radius of 200m. Wether we return routes that are very new or familiar to the user can be chosen by the user. The actual route distance and percentage of the part of the route that is unexplored is later displayed in the interface to the user. Also, by simply looking at the route on the activity map, one can clearly see what parts are in unexplored areas.\\
Those are the most important and interesting parts of the implementation, other more trivial parts have been omitted. The whole implementation can be inspected in the Explox Github repository\footnote{www.github.com/kvnmlr/explox}.

\section{Evaluation}
//TODO Ding


\section{Conclusion}
//TODO

\subsection{One paragraph about individual
contribution}
I think this should go into a separate document.

% BALANCE COLUMNS
\balance{}

% REFERENCES FORMAT
% References must be the same font size as other body text.
\bibliographystyle{SIGCHI-Reference-Format}
\bibliography{sample}

\end{document}

%%% Local Variables:
%%% mode: latex
%%% TeX-master: t
%%% End:
